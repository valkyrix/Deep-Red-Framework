% !TEX root=dissertationmain.tex
\chapter{Conclusions and Recommendations}
\label{conc}

\paragraph{}This chapter will summarize the thesis, provide future development recommendations and describe its impact on the field of machine learning for offensive security.

\paragraph{}The machine learning field of study is continually evolving with deep learning being at the forefront of the field, providing levels of accuracy and efficiency previously unobtainable due to the lack of available processing power. This paper uses classification algorithms including, but not limited to, K-means clustering with silhouette and gap statistic values, as well as data statistic calculations such as covariance matrices. There has been much research carried out for machine learning in the blue team field, however, as highlighted in the literature review \ref{litreview}, there have been \textit{very few} research papers dedicated to red teaming or offensive security. The primary goal of this thesis was to improve this balance by creating grounds for more research and development on the red team side of security, as well as develop an application tool for example usage in this field. This goal was achieved as described via the following paragraphs.

\paragraph{}This thesis documents the design, development and testing of a proof of concept red teaming application, written in Python. The application, in brief, uses machine learning and common network scanning tools to provide the user with advanced information about a network. The application's primary function is to apply clustering techniques on the data from Nmap (open source tool) and Nessus (developed by Tenable\textsuperscript{TM}) scanner outputs (both together or individually) in several different approaches, determining vulnerable hosts on the network, based on probability and feature similarity. The application includes manual, assisted and automatic modes for each type of tool input, then provides results in both text and, if selected, graphical format. The application is very versatile and, by design, can be completely configured to adapt to any type of network environment using several parameters from \ref{usage}.  An example of the resulting output from the application during the testing stage can be found in \ref{hacklab}. These results include in depth information about each cluster as well as graphs displaying the clusterings. The application is able to determine possible vulnerable hosts based on several factors described in \ref{CMFV}, with a bias towards the Nessus vulnerabilities analysed in \ref{results_analysis}. As a tool designed for security professionals, does not include a graphical user interface, it is instead configured and executed through a single command. Subsequently to execution, the application writes two files to its local directory as follows. A dot file consisting of the primary clustering to allow it to be imported into other commercial graphing applications such as Gephi. Along with a text file consisting of the targets from the vulnerability analysis, which can be imported into exploitation tools such as Metasploit Framework from Offensive Security\textsuperscript{\textregistered} and Hydra, a cracking suite developed by THC.

\paragraph{}There is however, an issue in determining the vulnerabilities, which occurs when the application uses data from hosts with a majority encompassing the same vulnerability. In this case the resulting output of the combined function in dual mode will not show these hosts based off that specific vulnerability, in other words, it is possible that the application would not depend on that vulnerability for analysis. This is due the clustering algorithm by design, as it detects and highlights the differences in the hosts then displays the most different of them all in extra to the full clusterings. This issue and solutions have been discussed and analysed in \ref{criteval}. This can be resolved through future work, however, more future work recommendations based on this thesis are as follows.

\paragraph{}The application machine learning model may be reprogrammed to utilise neural networks or other advanced machine learning algorithms as appose to K-means to classify each host. This would increase the predictability and efficiency of the model although, careful considerations to the variety in network configurations as well as the computational power impact should be taken into account if this approach is used. An intuitive graphical user interface may be implemented on top of the application, which would benefit users with less confidence in the current command line interface. Currently the application returns the probability of the operating system detection due to time constraints, however, it would be more beneficial for the end user if the application returned the pre-detected vulnerabilities for each host as well as the chances of their existence on the live system. The current modular python implementation approach allows for easy modification of the core modules by technical users depending on the desired use, this has the unfortunate side effect of slow execution times due to the Python interpreter.  The application can be ported to other lower level languages such as C++ instead, allowing for a much faster execution time in exchange for lower customisability. Support for more network scanners may be implemented without much more development by adding extra parser classes and modifying the display class. This would make increase reliability and usability of the application in the case where the current scanners are too noisy or obstructive to be used on a certain network. Due to the proof of concept nature of the application in its current state, in addition to the improvements previously mentioned, further testing and polishing may be carried out to release the application for use in the field depending on its applied license agreement.

\paragraph{}In conclusion, this thesis has created grounds for further research and development into machine learning for offensive security by proving its effectiveness in the field. This thesis has also provided several examples for further development based on the internally developed application. 
