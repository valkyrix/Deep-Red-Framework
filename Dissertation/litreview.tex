% !TEX root = dissertationmain.tex
%Chapter
\chapter{Literature Review}
\label{litreview}

\paragraph{}This chapter will include a review of related works in machine learning and network security.

\paragraph{}Machine learning has been used in the information security industry ever since \cite{debar1992neural} created the first basic network traffic analyser to separate attacks from regular packet traffic using an artificial neural network. Then similarity measurements were used to find anomalies in inputted Unix commands sequences by \cite{lane1997application}. This lead to the creation of a cluster classifier by \cite{labib2002nsom} which analysed real-time traffic in a network and plotted it into a GUI graphing interface very similar functionally wise to the application created in this thesis. Like other blue teaming solutions that came before it, Labib’s solution allowed for anomalies such as malicious traffic to be detected automatically. Ever since the afore mentioned examples, blue team machine learning solutions were no longer few and far between. The industry has been continually proactive, developing new theories and solutions at an increasing rate. One of such recent developments include the article, Shallow and Deep network intrusion detection systems: A Taxonomy and Survey, \cite{hodo2017shallow}. This article tackles the issue of creating an efficient intrusion detection system to handle large scale data with changing patterns in real time. Another such example is \cite{niyaz2016deep}, where the authors discuss the optimal approach of using deep learning techniques for intrusion detection systems. All these developments, although great for the industry as a whole have been unbalanced with the majority of research lying strictly within the blue team. This has created a gap in research for the red team. Red teaming in its nature stresses the importance of balance between the two sides. The balance between these teams is important due to the simple fact that malicious individuals will not stop improving and developing new methods. Without the balance of research allowing red team to keep up with those of malicious intent, red teaming will risk falling too far behind and becoming obsolete. The following paragraphs will now focus on red teaming.

\paragraph{}\cite{ART}, developed the first machine learning red teaming software using decision tree algorithms. Lu Song named this software Auto Red Team and was successful in creating a semi-automated penetration testing platform. The first step of the framework involved capturing the traffic between the attacker and the victim machines whilst the attacking machine would launch every single exploit supported by the application. For this the framework used 42 exploits from the Metasploit command line edition. All 42 exploits which are hard programmed into the framework must first be executed against the victim manually at this initial learning stage. Each exploit is manually programmed with hard paths due to the fact that, in the year 2008, the Metasploit command line available although having much more than 42 exploits, did not include a query system to find the exploits efficiently. The entire network traffic for this first stage must be captured. 

\paragraph{}This captured traffic is then audited by the decision tree algorithm which will return a new composed attack strategy and forwarded to a new and improved second attacking unit. Before the strategy is sent to the new unit, it must be parsed and converted to Perl script as the composer output’s the strategies in a human readable format. This new attack unit would then execute this strategy automatically whilst capturing the traffic, however, upon completion it will ask the user to declare whether the exploit was successful or not in order to proceed. At this point the user’s decision and twelve metrics of data from each captured packet are used as training data from the decision tree, improving its efficiently and accuracy. If the user’s decision was that the exploit had failed, the framework will repeat the entire process until the user declares that the exploit was successful. The ART framework therefore uses a supervised learning version of a decision tree model. Among Lu Song’s future work recommendations is that the Perl converter be replaced with a C++ compiler which could translate the strategy composer output into executable code in real time. This would greatly reduce much of the processing time and convert the application model into a ‘just in time’ approach. Further, by simply using a recent version of Metasploit the framework would be able to use any exploit from a given library without the need to hard program the paths as variables, greatly increasing the efficiency in the initial stages and overall versatility of framework.

\paragraph{}This concludes the literature review chapter and the following chapter will review the thesis methodology.
