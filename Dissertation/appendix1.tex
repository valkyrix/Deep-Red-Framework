% !TEX root=dissertationmain.tex
\chapter{Code}
%chapter to contain all code references
\section{Requirements python file}
\label{requirementsfile}
\begin{lstlisting}[language=bash]
backports-abc==0.4
bokeh==0.12.1
certifi==2016.8.8
cycler==0.10.0
futures==3.0.5
Jinja2==2.8
MarkupSafe==0.23
matplotlib==1.5.1
numpy==1.11.1
pyparsing==2.1.8
python-dateutil==2.5.3
pytz==2016.6.1
PyYAML==3.11
requests==2.11.0
scikit-learn==0.17.1
scipy==0.18.0
singledispatch==3.4.0.3
six==1.10.0
tornado==4.4.1
tabulate==0.7.7
\end{lstlisting}

\section{Application Usage Parameters}
\label{usage}
The following includes the usage and parameter definitions for the application.
\begin{lstlisting}[language=bash]
usage: cluster.py [-h] [-s {manual,automatic,assisted}]
                  [-c {kmeans,dbscan,agglomerative}]
                  [--metric {euclidean,cosine,jaccard}] [-N] [-n N_CLUSTERS]
                  [-e EPSILON] [-m MIN_SAMPLES] [-cent] [-t] [-tp twinpath]
                  [-p] [-v]
                  path [path ...]
Cluster NMap/Nessus Output
positional arguments:
  path                  Paths to files or directories to scan
optional arguments:
  -h, --help            show this help message and exit
  -s {manual,automatic,assisted}, --strategy {manual,automatic,assisted}
  -c {kmeans,dbscan,agglomerative}, --method {kmeans,dbscan,agglomerative}
  --metric {euclidean,cosine,jaccard}
  -N, --nessus          use .nessus file input
  -n N_CLUSTERS, --n_clusters N_CLUSTERS
                        Number of kmeans clusters to aim for
  -e EPSILON, --epsilon EPSILON
                        DBSCAN Epsilon
  -m MIN_SAMPLES, --min_samples MIN_SAMPLES
                        DBSCAN Minimum Samples
  -cent, --centroids    plot only centroids graph, requires the use of "-p"
  -t, --twin            use both input formats to calculate vulnerable single
                        clusters, use with -tp and -N
  -tp twinpath, --twinpath twinpath
                        path to nmap xml if using twin clustering
  -p, --plot            Plot clusters on 2D plane
  -v, --verbosity       increase output verbosity
\end{lstlisting}

\section{Initialization Class - cluster.py}
\label{cluster.py}
%primary class used to launch the application with the majority consisting of calls to other classes and methods
\paragraph{}The following contains the raw code for the primary initializer class of file  'cluster.py' in the application parent folder. It is used to start the application using parameters from the usage, in Appendix \ref{usage}.  The code has been thuroughly commented with the intention of being modified by a potential tester for individual operational requirements. This is further enforced by the application being entirely modular with each module able to be modified without harming the others. The cluster subclass has been removed from this code and placed in \ref{cluster_subclass} in order to promote the legibility of this appendix thus must not be executed direction without concatination with the afore mentioned subclass first.
\begin{lstlisting}[language=python]
import logging

from sklearn.preprocessing import normalize

from clusterer_parts.optimal_k_k_means import optimalK
from clusterer_parts.analysis import get_common_features_from_cluster, get_common_feature_stats
from clusterer_parts.clustering import cluster_with_dbscan, cluster_with_kmeans, precompute_distances, \
    cluster_with_agglomerative, cluster_interactive, get_centroids, cluster_single_kmeans, get_k
from clusterer_parts.display import print_cluster_details, generate_dot_graph_for_gephi, create_plot, \
    create_plot_centroids, create_plot_only_centroids, twin, remove_large_clusters
from clusterer_parts.optimizing import sort_items_by_multiple_keys
from clusterer_parts.reduction import pca
from clusterer_parts.validation import validate_clusters, get_average_distance_per_cluster
from clusterer_parts.vectorize import vectorize
import numpy as np
from tabulate import tabulate
firstpass = True

if __name__ == "__main__":
    import argparse

    parser = argparse.ArgumentParser(description=u'Cluster NMap/Nessus Output')

    parser.add_argument('path', metavar='path', type=str, nargs='+', default=None,
                        help="Paths to files or directories to scan")

    parser.add_argument('-s', '--strategy', default="automatic", choices=["manual", "automatic", "assisted"])
    parser.add_argument('-c', '--method', default="kmeans", choices=["kmeans", "dbscan", "agglomerative"])
    parser.add_argument('--metric', default="euclidean", choices=["euclidean", "cosine", "jaccard"])
    parser.add_argument('-N', '--nessus', default="false", required=False, action='store_true',
                        help='use .nessus file input')

    parser.add_argument('-n', '--n_clusters', type=int, default=2, help='Number of kmeans clusters to aim for')
    parser.add_argument('-e', '--epsilon', type=float, default=0.5, help='DBSCAN Epsilon')
    parser.add_argument('-m', '--min_samples', type=int, default=5, help='DBSCAN Minimum Samples')
    parser.add_argument('-cent', '--centroids', default=False, required=False, action='store_true',
                        help='plot only centroids graph, requires the use of "-p"')
    parser.add_argument('-t', '--twin', default=False, required=False, action='store_true',
                        help='use both input formats to calculate vulnerable single clusters, use with -tp and -N')
    parser.add_argument('-tp', '--twinpath', metavar='twinpath', type=str, required=False,
                        help='path to nmap xml if using twin clustering')

    parser.add_argument('-p', '--plot', default=False, required=False, action='store_true',
                        help='Plot clusters on 2D plane')

    parser.add_argument("-v", "--verbosity", action="count", help="increase output verbosity")
    args = parser.parse_args()

    logging.basicConfig(format='\%(asctime)s \%(process)s \%(module)s \%(funcName)s \%(levelname)-8s :\%(message)s',
                        datefmt='\%m-\%d \%H:\%M')

    if args.verbosity == 1:
        logging.getLogger().setLevel(logging.INFO)
    elif args.verbosity > 1:
        logging.getLogger().setLevel(logging.DEBUG)

    if (args.twin == False):

        # Vectorize our input
        logging.info("Vectorizing Stage")
        vector_names, vectors, vectorizer = vectorize(args.path, args.nessus)
        logging.debug("Loaded {0} vectors with {1} features".format(len(vector_names), vectors.shape[1]))
        logging.info("Vectorizing complete")

        # normalise vectors first before passing them through PCA. PCA uses 2 dimensions
        logging.info("Normalising the vectors")
        normalized_vectors = normalize(vectors)
        logging.info("Reducing vectors to two dimensions with PCA")
        reduced_vectors = pca(normalized_vectors)
        logging.debug(
            "reduced to {0} vectors with {1} dimensions".format((reduced_vectors.shape[0]), reduced_vectors.shape[1]))

        # Cluster the vectors
        logging.info("Clustering")
        labels = cluster(vector_names, vectors, reduced_vectors, normalized_vectors, vectorizer, args.strategy,
                         args.method, args.n_clusters, args.epsilon, args.min_samples, args.metric)
        logging.info("Clustering Complete")
        # Test cluster validity
        overall_score, per_cluster_score = validate_clusters(vectors, labels)

        # Analysis relevant to the person reading results
        universal_positive_features, universal_negative_features, shared_features = get_common_features_from_cluster(
            vectors, labels, vectorizer)

        # logging.debug("Shared features: {0}".format(shared_features))

        # Reduce results and relevant information to per cluster data
        cluster_details = {}
        for cluster_id in per_cluster_score.keys():
            cluster_details[cluster_id] = {
                "silhouette": per_cluster_score[cluster_id],
                "shared_positive_features": shared_features[cluster_id]['positive'],
                # "shared_negative_features": shared_features[cluster_id]['negative'],
                "ips": [vector_names[x] for x in xrange(len(vector_names)) if labels[x] == cluster_id]
            }
        print "Note: shared features does not retain keys from XML and therefore wont always be human readable."
        print_cluster_details(cluster_details, shared_features)

        if args.plot:
            # only kmeans centroids for now
            if no_clusters.startswith("kmeans") :
                logging.debug("Getting centroids using reduced vectors:")
                # global centroidskmeans
                # take just cluster number from result string
                n_clusters = no_clusters.split("=", 1)[1]
                n_clusters = int(n_clusters.rsplit(')', 1)[0])
                logging.debug("nclusters: " + str(n_clusters))

                centroidskmeans = get_centroids(reduced_vectors, n_clusters)
                logging.debug("attempting to plot the following centroids:\n " + str(centroidskmeans))

                # covariance
                x = centroidskmeans[:, 0]
                y = centroidskmeans[:, 1]
                X = np.vstack((x, y))
                cov = np.cov(X)
                logging.info("Centroids Covariance Matrix:\n {0}".format(cov))

                # print similarity distance between centroids
                matrix = precompute_distances(centroidskmeans, metric=args.metric)
                matrixTable = tabulate(matrix)
                logging.info(
                    "distance matrix between centroids using metric: {0} :\n{1}".format(args.metric, matrixTable))

                if args.centroids:
                    create_plot_only_centroids(reduced_vectors, labels, vector_names, centroidskmeans, n_clusters)
                else:
                    create_plot_centroids(reduced_vectors, labels, vector_names, centroidskmeans, n_clusters,
                                          cluster_details)

            # manually selected kmeans though arguments
            elif args.method == "kmeans" and args.strategy != "automatic":
                if get_k()>0:
                    centroidskmeans = get_centroids(reduced_vectors, get_k())
                else:
                    centroidskmeans = get_centroids(reduced_vectors, args.n_clusters)

                logging.debug("attempting to plot the following centroids: \n" + str(centroidskmeans))

                # covariance
                x = centroidskmeans[:, 0]
                y = centroidskmeans[:, 1]
                X = np.vstack((x, y))
                cov = np.cov(X)
                logging.info("Centroids Covariance Matrix:\n {0}".format(cov))

                # print similarity distance between centroids
                matrix = precompute_distances(centroidskmeans, metric=args.metric)
                matrixTable = tabulate(matrix)
                logging.info(
                    "distance matrix between centroids using metric: {0} :\n{1}".format(args.metric, matrixTable))

                if args.centroids:
                    if get_k() > 0:
                        create_plot_only_centroids(reduced_vectors, labels, vector_names, centroidskmeans, get_k())
                    else:
                        create_plot_only_centroids(reduced_vectors, labels, vector_names, centroidskmeans, args.n_clusters)
                else:
                    if get_k() > 0:
                        create_plot_centroids(reduced_vectors, labels, vector_names, centroidskmeans, get_k(),
                                              cluster_details)
                    else:
                        create_plot_centroids(reduced_vectors, labels, vector_names, centroidskmeans, args.n_clusters,
                                              cluster_details)
            else:
                logging.debug("plotting standard graph")
                create_plot(reduced_vectors, labels, vector_names)
        # Write DOT diagram out to cluster.dot, designed for input into Gephi (https://gephi.org/)
        with open("cluster.dot", "w") as f:
            f.write(
                generate_dot_graph_for_gephi(precompute_distances(vectors, metric=args.metric), vector_names, labels))

    elif args.twin == True and args.strategy == "automatic":

        logging.debug("twin flag enabled")
        logging.debug("tp: {0} , path: {1}".format(args.twinpath, args.path))

        # Vectorize our input for nessus
        logging.info("Vectorizing Stage for Nessus")
        Nvector_names, Nvectors, Nvectorizer = vectorize(args.path, args.nessus)
        logging.debug("Loaded {0} vectors with {1} features".format(len(Nvector_names), Nvectors.shape[1]))
        logging.info("Vectorizing complete\n")

        # Vectorize our input for nmap
        logging.info("Vectorizing Stage for nmap")
        twinpath = list()
        twinpath.append(args.twinpath)
        vector_names, vectors, vectorizer = vectorize(twinpath, False)
        logging.debug("Loaded {0} vectors with {1} features".format(len(vector_names), vectors.shape[1]))
        logging.info("Vectorizing complete\n")

        # normalise vectors first before passing them through PCA. PCA uses 2 dimensions
        # nessus
        logging.info("Normalising the nessus vectors")
        Nnormalized_vectors = normalize(Nvectors)
        logging.info("Reducing vectors to two dimensions with PCA")
        Nreduced_vectors = pca(Nnormalized_vectors)
        logging.debug(
            "reduced to {0} vectors with {1} dimensions".format((Nreduced_vectors.shape[0]), Nreduced_vectors.shape[1]))
        logging.info("Normalising complete\n")

        # normalise vectors first before passing them through PCA. PCA uses 2 dimensions
        # nmap
        logging.info("Normalising the nmap vectors")
        normalized_vectors = normalize(vectors)
        logging.info("Reducing vectors to two dimensions with PCA")
        reduced_vectors = pca(normalized_vectors)
        logging.debug(
            "reduced to {0} vectors with {1} dimensions".format((reduced_vectors.shape[0]), reduced_vectors.shape[1]))
        logging.info("Normalising complete\n")

        # Cluster the vectors
        logging.info("Clustering Nessus")
        Nlabels = cluster(Nvector_names, Nvectors, Nreduced_vectors, Nnormalized_vectors, Nvectorizer, args.strategy,
                          args.method, args.n_clusters, args.epsilon, args.min_samples, args.metric)
        logging.info("Clustering Complete\n\n")
        # Test cluster validity
        Noverall_score, Nper_cluster_score = validate_clusters(Nvectors, Nlabels)

        # Cluster the vectors
        logging.info("Clustering Nmap")
        labels = cluster(vector_names, vectors, reduced_vectors, normalized_vectors, vectorizer, args.strategy,
                         args.method, args.n_clusters, args.epsilon, args.min_samples, args.metric)
        logging.info("Clustering Complete\n\n")
        # Test cluster validity
        overall_score, per_cluster_score = validate_clusters(vectors, labels)

        # Analysis relevant to the person reading results
        # nessus
        Nuniversal_positive_features, Nuniversal_negative_features, Nshared_features = get_common_features_from_cluster(
            Nvectors, Nlabels, Nvectorizer)

        # Analysis relevant to the person reading results
        # nmap
        universal_positive_features, universal_negative_features, shared_features = get_common_features_from_cluster(
            vectors, labels, vectorizer)

        # Reduce results and relevant information to per cluster data
        # nessus
        Ncluster_details = {}
        for cluster_id in Nper_cluster_score.keys():
            Ncluster_details[cluster_id] = {
                "silhouette": Nper_cluster_score[cluster_id],
                "shared_positive_features": Nshared_features[cluster_id]['positive'],
                "ips": [Nvector_names[x] for x in xrange(len(Nvector_names)) if Nlabels[x] == cluster_id]
            }
        print "Note: shared features does not retain keys from XML and therefore wont always be human readable."
        print "Printing Nessus cluster details\n"
        print_cluster_details(Ncluster_details, Nshared_features)

        print "\n\n"

        # Reduce results and relevant information to per cluster data
        cluster_details = {}
        for cluster_id in per_cluster_score.keys():
            cluster_details[cluster_id] = {
                "silhouette": per_cluster_score[cluster_id],
                "shared_positive_features": shared_features[cluster_id]['positive'],
                # "shared_negative_features": shared_features[cluster_id]['negative'],
                "ips": [vector_names[x] for x in xrange(len(vector_names)) if labels[x] == cluster_id]
            }
        print "Printing Nmap cluster details\n"
        print_cluster_details(cluster_details, shared_features)

        if args.plot:
                # Nmap
                logging.debug("Getting centroids using reduced vectors for Nmap:")
                # take just cluster number from result string
                n_clusters = Nno_clusters.split("=", 1)[1]
                n_clusters = int(n_clusters.rsplit(')', 1)[0])
                logging.debug("nclusters: " + str(n_clusters))

                centroidskmeans = get_centroids(reduced_vectors, n_clusters)
                k = get_k()
                logging.debug("attempting to plot the following centroids:\n " + str(centroidskmeans) + "\n\n")

                # Nessus
                logging.debug("Getting centroids using reduced vectors for Nessus:")
                # take just cluster number from result string
                logging.debug("nclusters: " + str(Nno_clusters))

                Nn_clusters = no_clusters.split("=", 1)[1]
                Nn_clusters = int(Nn_clusters.rsplit(')', 1)[0])
                logging.debug("nclusters: " + str(Nn_clusters))

                Ncentroidskmeans = get_centroids(Nreduced_vectors, Nn_clusters)
                logging.debug("attempting to plot the following centroids:\n " + str(Ncentroidskmeans) + "\n\n")
                Nk = get_k()

                # covariance for Nmap
                x = centroidskmeans[:, 0]
                y = centroidskmeans[:, 1]
                X = np.vstack((x, y))
                cov = np.cov(X)
                logging.info("Nmap Centroids Covariance Matrix:\n {0}".format(cov))

                # covariance for Nessus
                Nx = Ncentroidskmeans[:, 0]
                Ny = Ncentroidskmeans[:, 1]
                NX = np.vstack((Nx, Ny))
                Ncov = np.cov(NX)
                logging.info("Nessus Centroids Covariance Matrix:\n {0}".format(Ncov))

                # print similarity distance between centroids
                # Nessus
                matrix = precompute_distances(centroidskmeans, metric=args.metric)
                matrixTable = tabulate(matrix)
                logging.info(
                    "distance matrix between centroids using metric for Nmap: {0} :\n{1}".format(args.metric, matrixTable))

                # print similarity distance between centroids
                # Nmap
                Nmatrix = precompute_distances(Ncentroidskmeans, metric=args.metric)
                NmatrixTable = tabulate(Nmatrix)
                logging.info(
                    "distance matrix between centroids using metric for Nessus: {0} :\n{1}".format(args.metric, NmatrixTable))

                small_ips = remove_large_clusters()

                logging.info("IP's from clusters with less than 3 IP's:\n {0}".format((small_ips)))


                #creates large array with 2nd dimension as large enough to hold both feature vectors
                nesmap = np.zeros((len(small_ips), (Nvectors.shape[1]+vectors.shape[1])))


                #for each single ip
                for index in range(len(small_ips)):
                    # for each ip in vectors
                    features = 0
                    for index2 in range(vectors.shape[0]):
                        #if ip is equal to vector ip
                        #logging.debug("if {0} = {1} ".format(small_ips[index], vector_names[index2]))
                        if small_ips[index] == vector_names[index2]:
                            #logging.debug("ip is equal to vector ip")
                            #for every one of this vectors features
                            for index3 in range(vectors.shape[1]):
                                #assign its features to single ip vector
                                nesmap[index,features] = vectors[index2, index3]
                                features +=1
                            break

                    for index2 in range(Nvectors.shape[0]):
                        #logging.debug("if {0} = {1} ".format(small_ips[index], Nvector_names[index2]))
                        if small_ips[index] == Nvector_names[index2]:
                            #logging.debug("ip is equal to vector ip")
                            for index3 in range(Nvectors.shape[1]):
                                #append nessus features onto nmap features
                                nesmap[index,features] = Nvectors[index2, index3]
                                features += 1
                            break

                logging.debug("Loaded {0} vectors with {1} features".format(nesmap.shape[0], nesmap.shape[1]))
                small_normalized_vectors = normalize(nesmap)
                logging.info("Normalizing input and reducing vectors to two dimensions with PCA")
                final = pca(small_normalized_vectors)

                logging.info("Resulting single IP vectors:\n {0}".format(final))

                Smatrix = precompute_distances(final, metric=args.metric)
                SmatrixTable = tabulate(Smatrix)
                logging.info(
                    "distance matrix between centroids of small combined clusters: {0} :\n{1}".format(args.metric, SmatrixTable))
                clusterz = cluster_single_kmeans(final, 2)

                logging.info("Writing recommended attack IP's to targets.txt for exploitation\n {0}")
                f = open('targets.txt', 'w')
                for index in range(len(small_ips)):
                    f.write('{0}\n'.format(small_ips[index]))  # python will convert \n to os.linesep
                f.close()  # you can omit in most cases as the destructor will call it

                twin(reduced_vectors, labels, vector_names, centroidskmeans, n_clusters, cluster_details, Nreduced_vectors, Nlabels, Nvector_names, Ncentroidskmeans, Nn_clusters, Ncluster_details, small_ips, final, clusterz, twinpath)

    else: print "not yet implemented #todo"
\end{lstlisting}


\section{Vectorization Class - vectorize.py}
\label{vectorize.py}
%raw vectorization class from clusterer_parts
\paragraph{}The following contains the raw code for the vectorize.py class. This class is called by the main cluster.py class at appendix \ref{cluster.py} in order to extract the information from the scan files as well as vectorize that information to be returned to the primary class.
\begin{lstlisting}[language=python]
import logging
from parsing import parsers
from parse_nessus import Nparsers
from single_ip_parsing_nmap import single_ip_parsers
import numpy as np


class Vectorizer:
    """
    This class handles the vectorizing of input
    It additionally stores all pseudo vectors until we are ready for the finished vectors
    """

    def __init__(self):
        self.tokenized_strings = []
        self.pseudo_vectors = {}

    def add_string_to_ip(self, ip, string):
        if ip not in self.pseudo_vectors:
            self.pseudo_vectors[ip] = []

        if string not in self.tokenized_strings:
            self.tokenized_strings.append(string)

        s_id = self.tokenized_strings.index(string)
        self.pseudo_vectors[ip].append(s_id)

    def parse_input(self, input_string, n):
    ##n boolean value refers to nessus input only
        if (n==True):
            for parser in Nparsers:
                if parser.can_parse_input(input_string):
                    results = parser.parse_input(input_string)
                    for key in results.keys():
                        for s in results[key]:
                            self.add_string_to_ip(key, s)
        else:
            for parser in parsers:
                if parser.can_parse_input(input_string):
                    results = parser.parse_input(input_string)
                    for key in results.keys():
                        for s in results[key]:
                            self.add_string_to_ip(key, s)

    def output_vectors(self):
        vector_names = []
        vectors = np.zeros((len(self.pseudo_vectors.keys()), len(self.tokenized_strings)), dtype=np.float)

        for ip_index, ip in enumerate(self.pseudo_vectors.keys()):
            vector_names.append(ip)
            for s_index in self.pseudo_vectors[ip]:
                # Just set it to one, we want to ignore any case we see a value more than once
                vectors[ip_index, s_index] = 1

        return vector_names, vectors


def vectorize(files_to_vectorize, n):
    vectorizer = Vectorizer()
    for file_path in files_to_vectorize:
        with open(file_path, "r") as f:
            vectorizer.parse_input(f.read(), n)

    vector_names, vectors = vectorizer.output_vectors()

    return vector_names, vectors, vectorizer

def parse_single_ips(files_to_vectorize, ips):
    for file_path in files_to_vectorize:
        with open(file_path, "r") as f:
            for parser in single_ip_parsers:
                #logging.debug("vectorisor selecting single ips")
                results = parser.parse_input(f.read(), ips)
                return results
\end{lstlisting}

\section{Clustering Algorithm subclass}
\label{cluster_subclass}
\paragraph{}The following code consists of the clustering algorithm used for all the modes within the application. It uses the provided parameters to decide between these modes as well as calls several other subclasses for code modularity.
\begin{lstlisting}[language=python]
def cluster(
        vector_names,
        vectors,
        reduced_vectors,
        normalized_vectors,
        vectorizer,
        strategy="automatic",
        cluster_method="kmeans",
        n_clusters=2,
        epsilon=0.5,
        min_samples=5,
        metric="euclidean",
):
    """
    Clustering options:

    Manual:
     The user supplies all required information to do the clustering. This includes the clustering algorithm and
     hyper parameters,
     if no cluster count is provided the gap_statistic method will be used to calculate the optimal cluster count

    Assisted:
     The user assists the algorithm by suggesting that some samples should or should not be clustered together

    Automatic:
     The multiple clustering strategies and parameters are used in an attempt to get the best clusters
     
     finds the least amount of clusters with atleast one shared feature
     
     only uses gap statistic for small IP clusters
    """

    global centroidskmeans, centroidagglo, centroiddbs, no_clusters, Nno_clusters, Ncentroidskmeans



    if strategy == "manual":
        no_clusters = ""
        if cluster_method == "kmeans":
            #centroidskmeans = get_centroids(reduced_vectors, n_clusters=n_clusters)
            #logging.debug("centroids for kmeans: {0}".format(centroidskmeans))
            k, gapdf = optimalK(vectors, nrefs=3, maxClusters=reduced_vectors.shape[0])
            return cluster_with_kmeans(reduced_vectors, n_clusters=k)

        elif cluster_method == "dbscan":
            return cluster_with_dbscan(reduced_vectors, epsilon=epsilon, min_samples=min_samples, metric=metric)

        elif cluster_method == "agglomerative":
            return cluster_with_agglomerative(reduced_vectors, n_clusters=n_clusters, metric=metric)

        else:
            # Unknown clustering method
            raise NotImplementedError()

    elif strategy == "assisted":
        """
        To display a information about a vector to a user, you can use the following:
        display_vector_index_details(vector_index, vectors, vector_names, vectorizer)
        """

        return cluster_interactive(reduced_vectors, vectorizer, vectors, vector_names)
    elif strategy == "automatic":
        results = []
        smallest_cluster_count = vectors.shape[0]
        # centroids works for only kmeans atm
        for cluster_method in [
            #todo add agglo and dbscan back in after they can return centroids.
            "kmeans"  # ,
            # "agglomerative",
            # "dbscan",
        ]:
            if cluster_method == "kmeans":
                #this method is called X-means clustering
                logging.debug("Starting prospective KMeans clusterings")
                move_to_next_method = False
                # start at 2 clusters and end at smallest_cluster_count
                for n_clusters in xrange(2, smallest_cluster_count):
                    logging.debug("Trying {0}".format("kmeans(n_clusters={0})".format(n_clusters)))
                    labels = cluster_with_kmeans(reduced_vectors, n_clusters=n_clusters)
                    overall_score, per_cluster_score = validate_clusters(vectors, labels)
                    mean_distance = get_average_distance_per_cluster(vectors, labels)[0]

                    tsp, msp, msn = get_common_feature_stats(vectors, labels, vectorizer)

                    # If any cluster has 0 shared features, we just ignore the result
                    if msp <= tsp:
                        logging.debug("Not all clusters are informative (a cluster has 0 shared features) ")
                        continue
                    if len(set(labels)) > smallest_cluster_count:
                        move_to_next_method = True
                        # logging.debug("len(set(labels)): {0} > smallest_cluster_count: {1}".format(len(set(labels)), smallest_cluster_count))
                        break
                    if len(set(labels)) < smallest_cluster_count:
                        smallest_cluster_count = len(set(labels))
                    #too verbose
                    # logging.debug(repr((
                    #         overall_score,
                    #         min(per_cluster_score.values()),
                    #         mean_distance,
                    #         labels,
                    #         len(set(labels)),
                    #         tsp,
                    #         msp,
                    #         msn,
                    #         "kmeans(n_clusters={0})".format(n_clusters)
                    #     )))
                    results.append(
                        (
                            overall_score,
                            min(per_cluster_score.values()),
                            mean_distance,
                            labels,
                            len(set(labels)),
                            tsp,
                            msp,
                            msn,
                            "kmeans(n_clusters={0})".format(n_clusters)
                        )
                    )
                if move_to_next_method:
                    continue

            if cluster_method == "agglomerative":
                logging.debug("Starting prospective Agglomerative clusterings")
                move_to_next_method = False
                for n_clusters in xrange(2, smallest_cluster_count):
                    logging.debug("Trying {0}".format("agglomerative(n_clusters={0})".format(n_clusters)))
                    labels = cluster_with_agglomerative(reduced_vectors, n_clusters=n_clusters, metric=metric)
                    overall_score, per_cluster_score = validate_clusters(vectors, labels)
                    mean_distance = get_average_distance_per_cluster(vectors, labels)[0]

                    tsp, msp, msn = get_common_feature_stats(vectors, labels, vectorizer)

                    # If any cluster has 0 shared features, we just ignore the result
                    if msp <= tsp:
                        logging.debug("Not all clusters are informative (a cluster has 0 shared features) ")
                        continue
                    if len(set(labels)) > smallest_cluster_count:
                        move_to_next_method = True
                        break
                    if len(set(labels)) < smallest_cluster_count:
                        smallest_cluster_count = len(set(labels))

                    logging.debug(repr((
                        overall_score,
                        min(per_cluster_score.values()),
                        mean_distance,
                        labels,
                        len(set(labels)),
                        tsp,
                        msp,
                        msn,
                        "agglomerative(n_clusters={0})".format(n_clusters)
                    )))
                    results.append(
                        (
                            overall_score,
                            min(per_cluster_score.values()),
                            mean_distance,
                            labels,
                            len(set(labels)),
                            tsp,
                            msp,
                            msn,
                            "agglomerative(n_clusters={0})".format(n_clusters)
                        )
                    )
                if move_to_next_method:
                    continue

            if cluster_method == "dbscan":
                logging.debug("Starting prospective DBSCAN clusterings")
                distance_matrix = precompute_distances(vectors, metric=metric)
                min_distance = sorted(set(list(distance_matrix.flatten())))[1]
                max_distance = sorted(set(list(distance_matrix.flatten())))[-1]
                num_steps = 25.0
                step_size = float(max_distance - min_distance) / float(num_steps)
                epsilon = min_distance
                while True:
                    logging.debug("Trying {0}".format("dbscan(epsilon={0})".format(epsilon)))
                    labels = cluster_with_dbscan(reduced_vectors, epsilon=epsilon, min_samples=1,
                                                 distances=distance_matrix)
                    if len(set(labels)) == 1 and list(set(labels))[0] == 0:
                        break
                    overall_score, per_cluster_score = validate_clusters(vectors, labels)
                    mean_distance = get_average_distance_per_cluster(vectors, labels)[0]

                    tsp, msp, msn = get_common_feature_stats(vectors, labels, vectorizer)

                    # If any cluster has 0 shared features, we just ignore the result
                    if msp <= tsp:
                        logging.debug("Not all clusters are informative (a cluster has 0 shared features) ")
                        epsilon += step_size
                        continue

                    logging.debug(repr((
                        overall_score,
                        min(per_cluster_score.values()),
                        mean_distance,
                        labels,
                        len(set(labels)),
                        tsp,
                        msp,
                        msn,
                        "dbscan(epsilon={0})".format(epsilon)
                    )))
                    results.append(
                        (
                            overall_score,
                            min(per_cluster_score.values()),
                            mean_distance,
                            labels,
                            len(set(labels)),
                            tsp,
                            msp,
                            msn,
                            "dbscan(epsilon={0})".format(epsilon)
                        )
                    )
                    epsilon += step_size

        # Choose best clustering result based on the following attributes
        sorted_results = sort_items_by_multiple_keys(
            results,
            {
                # 0: True,  # AVG Silhouette
                # 1: True,  # Min Silhouette
                # 2: False,  # Average distance
                4: False,  # Number of clusters
                # 6: True,   # Min common features per cluster
            },
            {
                # 0: 1,
                # 1: 1,
                # 2: 1,
                4: 1,
                # 6: 1
            }
        )
        # logging.debug(sorted_results)
        best_result = results[sorted_results[0][0]]
        # logging.debug(best_result)

        best_method = best_result[-1]
        best_silhouette = best_result[0]
        best_labels = best_result[3]
        global firstpass
        if firstpass:
            no_clusters = best_result[-1]
            firstpass = False
        else:
            Nno_clusters = best_result[-1]

        # no_clusters = best_result[-1]

        logging.info("Best clustering method: {0} (adjusted silhouette == {1})".format(best_method, best_silhouette))
        return best_labels

    else:
        # Unknown strategy
        raise NotImplementedError()
\end{lstlisting}

\section{Gap Statistic Implementation}
\label{gapstatistic}

\paragraph{}The following segment of codeis the defined function 'optimalK' which calculates the optimal value for K in K-means via the Gap Statistic method. This implementation is within the optimal\_k\_k\_means.py class and is thoroughly commentated.  The parameters and returned value are explained at the top of the function.

\begin{lstlisting}[language=python]

def optimalK(data, nrefs, maxClusters):
    """
    Calculates KMeans optimal K using Gap Statistic from http://web.stanford.edu/~hastie/Papers/gap.pdf
    Params:
        data: ndarry of shape (n_samples, n_features)
        nrefs: number of sample reference datasets to create
        maxClusters: Maximum number of clusters to test for
    Returns: (gaps, optimalK)
    """
    gaps = np.zeros((len(range(1, maxClusters)),))
    resultsdf = pd.DataFrame({'clusterCount': [], 'gap': []})
    for gap_index, k in enumerate(range(1, maxClusters)):

        # Holder for reference dispersion results
        refDisps = np.zeros(nrefs)

        # For n references, generate random sample and perform kmeans getting resulting dispersion of each loop
        for i in range(nrefs):
            # Create new random reference set
            randomReference = np.random.random_sample(size=data.shape)

            # Fit to it
            km = KMeans(k)
            km.fit(randomReference)

            refDisp = km.inertia_
            refDisps[i] = refDisp

        # Fit cluster to original data and create dispersion
        km = KMeans(k)
        km.fit(data)

        origDisp = km.inertia_

        # Calculate gap statistic
        gap = np.log(np.mean(refDisps)) - np.log(origDisp)

        # Assign this loop's gap statistic to gaps
        gaps[gap_index] = gap

        resultsdf = resultsdf.append({'clusterCount': k, 'gap': gap}, ignore_index=True)

    return (gaps.argmax() + 1, resultsdf)  # Plus 1 because index of 0 means 1 cluster is optimal, index 2 = 3 clusters are optimal
\end{lstlisting}